% Copyright (c) 2012-2025, Yegor Bugayenko
% All rights reserved.
%
% Redistribution and use in source and binary forms, with or without
% modification, are permitted provided that the following conditions
% are met: 1) Redistributions of source code must retain the above
% copyright notice, this list of conditions and the following
% disclaimer. 2) Redistributions in binary form must reproduce the above
% copyright notice, this list of conditions and the following
% disclaimer in the documentation and/or other materials provided
% with the distribution. 3) Neither the name of the s3auth.com nor
% the names of its contributors may be used to endorse or promote
% products derived from this software without specific prior written
% permission.
%
% THIS SOFTWARE IS PROVIDED BY THE COPYRIGHT HOLDERS AND CONTRIBUTORS
% "AS IS" AND ANY EXPRESS OR IMPLIED WARRANTIES, INCLUDING, BUT
% NOT LIMITED TO, THE IMPLIED WARRANTIES OF MERCHANTABILITY AND
% FITNESS FOR A PARTICULAR PURPOSE ARE DISCLAIMED. IN NO EVENT SHALL
% THE COPYRIGHT HOLDER OR CONTRIBUTORS BE LIABLE FOR ANY DIRECT,
% INDIRECT, INCIDENTAL, SPECIAL, EXEMPLARY, OR CONSEQUENTIAL DAMAGES
% (INCLUDING, BUT NOT LIMITED TO, PROCUREMENT OF SUBSTITUTE GOODS OR
% SERVICES; LOSS OF USE, DATA, OR PROFITS; OR BUSINESS INTERRUPTION)
% HOWEVER CAUSED AND ON ANY THEORY OF LIABILITY, WHETHER IN CONTRACT,
% STRICT LIABILITY, OR TORT (INCLUDING NEGLIGENCE OR OTHERWISE)
% ARISING IN ANY WAY OUT OF THE USE OF THIS SOFTWARE, EVEN IF ADVISED
% OF THE POSSIBILITY OF SUCH DAMAGE.

%%
%% SPDX-FileCopyrightText: Copyright (c) 2012-2025, Yegor Bugayenko
%% SPDX-License-Identifier: MIT
\tikzstyle{uml2} = [
    fill=rupBody,
    draw=rupBorder,
    font={\ttfamily},
]
\tikzstyle{uml2-class} = [ % class in class diagram
    uml2,
    rectangle split,
    rectangle split parts=3,
    % rectangle split every empty part={},% delete existing height, depth and width
    inner sep=0.1cm,
    rectangle split empty part height=0.1cm,
    text width=3cm,
    text centered,
    text justified,
]
\tikzstyle{uml2-object} = [ % instance/object in class diagram
    uml2,
    inner sep=0.1cm,
    text width=3cm,
    text centered,
    text justified,
]
\tikzstyle{uml2-package} = [ % package
    uml2,
    text centered,
    text justified,
]
\tikzstyle{uml2-comment} = [ % comment in class diagram
    font={\small\ttfamily},
    fill=white,
    midway,
    inner sep=0cm,
]
\tikzstyle{uml2-cardinality} = [ % cardinality
    font={\footnotesize\ttfamily}
]
%
\newcommand{\tikzpic}[1]{
    \tikz \node [scale=0.75] {\begin{tikzpicture}#1\end{tikzpicture}};
}
% [1]: name of component
% [2]: style, if necessary
\newcommand{\addComponentIco}[2]{
    \node[uml2, rectangle, below left=0.1cm of #1.north east,
        inner sep=0cm, minimum width=0.3cm, minimum height=0.3cm, #2] (#1-icon) {};
    \node[uml2, rectangle, below=0.05cm of #1-icon.north west,
        inner sep=0cm, minimum width=0.2cm, minimum height=0.075cm, #2]{};
    \node[uml2, rectangle, above=0.05cm of #1-icon.south west,
        inner sep=0cm, minimum width=0.2cm, minimum height=0.075cm, #2]{};
}
% [1]: name of artifact
% [2]: style, if necessary
\newcommand{\addArtifactIco}[2]{
    \node[uml2, rectangle, below left=0.1cm of #1.north east,
        inner sep=0cm, minimum width=0.3cm, minimum height=0.3cm, #2] (#1-icon) {};
}
% [1]: entity name
% [2]: title of the component
% [3]: the stereotype
% [4]: width of the text
% [5]: additional style, if necessary
\newcommand{\umlRectangle}[5]{
    \node [uml2, text width=#4, minimum height=1.2cm, #5] (#1) {
        \parbox{#4}{
            \centering
            {\color{gray}\small<{}<#3>{}>}\linebreak
            #2
        }
    };
}
% [1]: entity name
% [2]: title of the interface
% [3]: style
\newcommand{\umlInterface}[3]{
    \node [uml2, circle, minimum width=0.15cm, #3] (#1) {};
    \node [font={\ttfamily}, below=0mm of #1.south] {
        \parbox{2cm}{
            \centering
            #2
        }
    };
}
% [1]: name of the port to create
% [2]: center of the port
% [3]: style
\newcommand{\addPort}[3]{
    \node[uml2, rectangle,
        inner sep=0cm, minimum width=0.3cm, minimum height=0.3cm, #3] (#1)  at (#2) {};
}
% [1]: entity name
% [2]: title of the component
% [3]: width of the text
% [4]: additional style, if necessary
\newcommand{\umlNode}[4]{
    \node [inner sep=0cm, #4] (#1) {
        \begin{tikzpicture}
            \node [uml2, text width=#3, minimum height=1.2cm] (#1) {
                \parbox{#3}{
                    \centering
                    #2
                }
            };
            \path (#1.north west) +(0.2,0.2) coordinate (#1-cornerA);
            \path (#1.north east) +(0.2,0.2) coordinate (#1-cornerB);
            \path (#1.south east) +(0.2,0.2) coordinate (#1-cornerC);
            \path [uml2]
                (#1.north west)
                -- (#1-cornerA)
                -- (#1-cornerB)
                -- (#1-cornerC)
                -- (#1.south east)
                -- (#1.north east)
                -- cycle;
            \path [uml2] (#1.north east) -- (#1-cornerB);
        \end{tikzpicture}
    };
}
% #1 - TIKZ name of the element
% #2 - text to render
% #3 - coordinates/styles
\newcommand{\umlActor}[3]{
    \node [outer sep=-1mm, #3] (#1) {
        \tikz \draw [uml2, fill=white] (0,0) -- +(0,-0.5) % body
            -- +(-0.5,-1) % left leg
            +(0,-0.5) -- +(0.5,-1) % right leg
            +(-0.5,-0.1) -- +(0.5,-0.1) % hands
            +(0,+0.25) circle (0.25) %head
            +(0,-1) node [anchor=north] {#2};
    };
}
